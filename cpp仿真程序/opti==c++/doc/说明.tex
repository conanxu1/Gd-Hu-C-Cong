\documentclass[a4]{ctexart}



\begin{document}



\begin{table}[h]
\centering  % 显示位置为中间
\caption{类型的说明}  % 表格标题
\label{table1}  % 用于索引表格的标签
\begin{tabular}{|c|c|c|} 
\hline
Rn     &    $R^n$     &           向量类型  \\
\hline
dae\_f &   $R^n\times R^n\times R \mapsto R^n$ & 微分代数方程类型$f(x,u,t)$\\
 \hline
  opt\_int & $R^n\times R^n\times R\mapsto R^n$  & 拉格朗日目标型\\
 \hline
 opt\_phi   & $ R^n\times R\times R^n\times R\mapsto R^n$  & 拉格朗日目标型\\
 \hline
 Rn\_f   & $ R\mapsto R^n$  & 向量函数f(t)\\
 \hline
 
\end{tabular}
\end{table}



\begin{itemize}
\item 抽象类统一各种方程求解方法。
\item 优化问题中需要调用微分方程求解器和一般的优化器。
\end{itemize}







\section{求解器使用说明}
类Euler\_Ode\_Sol继承自DAE\_Solver

类DAE\_Solver计划设计一个代数微分方程求解器

类Euler\_Ode\_Sol是线性微分方程求解器
dx/dt=Ax+Bu

\begin{itemize}
\item 第一步Euler\_Ode\_Sol(dimx,dimu,t0,tf,分点数);//构造函数
\item 第二步 Euler\_Ode\_Sol.set(Rmn tA,Rmn tB,Rn\_f tu,Rn x0);//矩阵输入函数初值
\item 第三步Euler\_Ode\_Sol.sol( );//求解
\end{itemize}




double*  Legendre::P\_n(int n);

计算n次勒让德多项式的系数

double Poly\_Sub(double x,double* a\_n,int N);

秦九韶算法计算多项式的值












\end{document}