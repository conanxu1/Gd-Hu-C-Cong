\documentclass[11pt,a4paper]{ctexart}

\begin{document}
考虑系统
$$\dot{x}(t)=\sum_{i=1}^{n}A_ix(t-\tau_i)$$


\section{稳定性判别代码}

\begin{table}[hbt]
\caption{panding.m}  % 表格标题
	\label{table1}  % 用于索引表格的标签
	\begin{tabular}{|c|c|} 
\hline
函数原型& [W,pzend]=panding( Ai,taui,n  )\\
\hline
功能& 判断以$A_i$为系数,$\tau_i$为延时量的系统是否稳定\\
\hline
输入项& $Ai$为元胞数组形式;$\tau_i$为数组形式\\
\hline
输出项&W不稳定根个数,pzend为$P(z)$的轨迹\\
\hline
算法&通过离散半圆,计算$P(z)$.用matlab的angle函数计算幅角.\\
&并将幅角的变化量连续化.\\
\hline
\end{tabular}
\end{table}
注:
\begin{itemize}
\item 此程序最为需要注意的是,在$P(z)$比较接近$0$时 ,幅角的变化快,从而导致结果不准确.这涉及参考文章中提到的tolerance $\delta$.本程序在$0$附近对轨迹进行了加细.
\item $Ai\{N+1\}$是$A_0$;
\end{itemize}


参考文章
<DELAY-DEPENDENT STABILITY OF RUNGE–KUTTA
METHODS FOR LINEAR NEUTRAL SYSTEMS
WITH MULTIPLE DELAYS>



\newpage

\section{稳定性判别代码=v2}




\begin{table}[hbt]
\caption{numWv2.m}  % 表格标题
	\label{table1}  % 用于索引表格的标签
	\begin{tabular}{|c|c|} 
\hline
函数原型& [W,tnew,xnew] = numWv2( Ai,taui,N) \\
\hline
功能    & 判断以$A_i$为系数,$\tau_i$为延时量的系统是否稳定\\
\hline
输入项  & $Ai$为元胞数组形式;$\tau_i$为数组形式\\
\hline
输出项  & W为不稳定根个数, tnew(i)=$\theta_i$为$Im(P(s(\theta_i)))=0$   \\
& xnew(i)$= Im(P(s(\theta_i))),Im(P(s(\theta_i)))=0$  \\
\hline
算法    & 通过求根程序计算$Im(P(s(\theta_i)))=0$ , \\
        & $Im(P(s(\theta_i+a)) -P(s(\theta_i-a)))$的符号判断穿越x轴的方向 \\
        &计算轨迹与x轴正半轴穿越的总和得到W\\ 
\hline
\end{tabular}
\end{table}
注:
\begin{itemize}
\item 对高阶系统,本程序求根时的误差较大,需要具体分析.高阶系统的$Im(P(s(\theta_i)))$部分图像与x轴非常接近求根误差较大.
\end{itemize}












































\end{document}

